\section{Мандри - Орися}
\begin{guitar}

Ой н[Am]е ходи, ходи, Орисю, н[E]а гору, на Л[Am]иску.
Н[Am]е ходи, ходи, Орисю, a[E] грай на соп[Am]ілц[G7]і.

[C]Ой-гой, дриці-дриці, [G]oй-гой, дриці-дриці
Н[A]е ходи на Лису гору, [E]А грай на соп[Am]ілц[G7]і.
[C]Ой-гой, дриці-дриці, [G]oй-гой, дриці-дриці
Н[A]е ходи на Лису гору, [E]А грай на соп[Am]ілці.

Бо на горі, на Лисці туман над ярами.
Бо на горі, на Лисці чорти з відьмаками.

Ой-гой, дриці-дриці oй-гой, дриці-дриці
Там шугає така нечисть, що гріх і дивитись.

В чорта роги - круторогі, очі - як лещата.
Жінка в нього - чорна жаба, бридка та вусата.

Ой-гой, дриці-дриці oй-гой, дриці-дриці 
Отакі жінки в тім царстві, хвайні молодиці.

Чорт зубатий та багатий хоче кльову дівку,
Може їй платити златом та водить до шинку.

Ой-гой, дриці-дриці oй-гой, дриці-дриці 
Не ходи, Орись, на гору, a чекай на принця.
\end{guitar}
