\section{Стары Ольса - У карчме}
\begin{guitar}

(сярэднявечная застольная песня, пераклад з лацінскай 
З.Сасноўскага, А.Апановіча і А.Чумакова, незначна скарочаны варыянт)

У карчме калі мы былі
Справы нас не клапацілі
Ў галаве адны забавы
Піва пі вяселлю слава
Той, хто у карчме бывае
Тлушч у цела назбірае
Стане моцным і здаровым
Той, хто чуе гэты словы

Хто бярэцца з намі піці
Той нясумна будзе жыці
У забавах гора гіне
Зноў лье піва гаспадыня
Захлябнецца гора півам
Грукнуць келіхі шчасліва
Будзем піці, жартаваці
Будзем славіць Вакха, браце!

Першы тост звычайна п'ецца
За ўсіх тых, хто тут збярэцца
П'ем другі, каб волю меці
За жыццё падымем трэці
Чацьвёрты за Царкву Хрыстову
Пяты за моц князёву
Шосты за людзей вольных
А потым п'е, хто здольны

Паўтор 1.
\end{guitar}
