\section{Тартак - Лицарський хрест}
\begin{guitar}
\begin{flushleft}
\begin{tabbing}
Вступ: \=Gm D\guitarSharp\ Gm D\guitarSharp\ Gm D\guitarSharp\ D D\guitarSharp\ \\
       \>Gm\\
\end{tabbing}

Дихає ліс,
Пташка на г[D#]іллі
Пісню спів[Hb]ає, що тішить мій сл[D]ух.
Я довго ріс – 
Йшов через цілі,
Що тіло гартують і зміцнюють дух.
Тиха роса
Зіб'ється з трав
Криком "вперед!", дружним тупотом ніг.
Я тут знайшов
Те, що шукав,
Славу здобув і себе переміг!

Приспів:

Мій лицарський хр[Gm]ест – 
Моя нагор[D#]ода
За те, що не вп[Hm]ав, за те, що не вт[F]ік!
Мій лицарський хрест – 
Яскрава пригода,
Що буде тривати в мені цілий вік!
Мій лицарський хрест...

Вступ.

Плинуть роки,
Їх заметілі
Скроні мої пофарбують у сніг.
Я, завдяки
Шрамам на тілі,
В пам'ять свою закарбую усіх
Друзів моїх 
Та ворогів -
Кого любив і кого вбивав...
Може чогось
Я не зумів,
Та не згубив, не програв, не продав...

Приспів.

Тим, що загинули, й тим, що вижили,
Слово своє вдячно присвячую.
Хай ворог зиркає очима хижими,
Нехай гарчить – мені не лячно!
Хто вріс корінням, той не зламається.
Хто має стержень, той не зігнеться.
Любов до матері – найкраща порадниця.
Любов до вітчизни – ідея серця.
Лицарський хрест – відзнака для обраних – 
Не завжди на грудях, а в діях і звершеннях.
Для тих, що в боях ставали хоробрими.
Для тих, що в атаки здіймалися першими.
Не відступитися від слова сказаного – 
Дерти руками, зубами гризти.
У світі багато брудного й заразного,
Але той, хто хоче, залишається чистим!
\end{flushleft}

\end{guitar}